\documentclass[a4paper]{article}
% generated by Docutils <http://docutils.sourceforge.net/>
\usepackage{fixltx2e} % LaTeX patches, \textsubscript
\usepackage{cmap} % fix search and cut-and-paste in Acrobat
\usepackage{ifthen}
\usepackage[T1]{fontenc}
\usepackage[utf8]{inputenc}

%%% Custom LaTeX preamble
% PDF Standard Fonts
\usepackage{mathptmx} % Times
\usepackage[scaled=.90]{helvet}
\usepackage{courier}

%%% User specified packages and stylesheets

%%% Fallback definitions for Docutils-specific commands

% hyperlinks:
\ifthenelse{\isundefined{\hypersetup}}{
  \usepackage[colorlinks=true,linkcolor=blue,urlcolor=blue]{hyperref}
  \urlstyle{same} % normal text font (alternatives: tt, rm, sf)
}{}


%%% Body
\begin{document}
%
\begin{itemize}

\item Fibers

\end{itemize}

The beam into a fiber is f2.2 or so. Fibers scramble incoming light
azimuthally but not radially (I still don't understand why not -{}-
CPL), so the output from a fiber will be a symmetric ``cone'', barring
cracks/blockage/etc very near the exit. But the profile of that cone
does vary, depending on the illumination pattern on the entrance. The
PFS fiber plane is quite significantly vignetted (due to the size of
the front optics), and the illumination pattern on the front of a
fiber depends significantly on its distance from the center of the
instrument, as its view of the primary becomes obscured.

\end{document}
